%%%%%%%%%%%%%%%%%%%%%%%%%%%%%%%%%%%%%%%%%
% Arsclassica Article
% Structure Specification File
%
% This file has been downloaded from:
% http://www.LaTeXTemplates.com
%
% Original author:
% Lorenzo Pantieri (http://www.lorenzopantieri.net) with extensive modifications by:
% Vel (vel@latextemplates.com) %
% License:
% CC BY-NC-SA 3.0 (http://creativecommons.org/licenses/by-nc-sa/3.0/)
%
%%%%%%%%%%%%%%%%%%%%%%%%%%%%%%%%%%%%%%%%%

%----------------------------------------------------------------------------------------
%	REQUIRED PACKAGES
%----------------------------------------------------------------------------------------

\usepackage[
nochapters, % Turn off chapters since this is an article
beramono, % Use the Bera Mono font for monospaced text (\texttt)
%eulermath,% Use the Euler font for mathematics
pdfspacing, % Makes use of pdftex’ letter spacing capabilities via the microtype package
dottedtoc % Dotted lines leading to the page numbers in the table of contents
]{classicthesis} % The layout is based on the Classic Thesis style

\usepackage{arsclassica} % Modifies the Classic Thesis package

\usepackage{libertine}

\usepackage[T1]{fontenc} % Use 8-bit encoding that has 256 glyphs

\usepackage[utf8]{inputenc} % Required for including letters with accents

\usepackage{graphicx} % Required for including images
\graphicspath{{Figures/}} % Set the default folder for images

\usepackage{enumitem} % Required for manipulating the whitespace between and within lists

\usepackage{lipsum} % Used for inserting dummy 'Lorem ipsum' text into the template

\usepackage{subfig} % Required for creating figures with multiple parts (subfigures)

\usepackage{amsmath,amssymb,amsthm} % For including math equations, theorems, symbols, etc

\usepackage{varioref} % More descriptive referencing

\usepackage{mathtools} % for \coloneqq

\usepackage[normalem]{ulem} % for \sout

\usepackage{listings} % for listings
\usepackage{MnSymbol} % line break symbols for listings

%----------------------------------------------------------------------------------------
%	THEOREM STYLES
%---------------------------------------------------------------------------------------

\newtheorem{dummy}{Dummy}[section]

\theoremstyle{definition} % Define theorem styles here based on the definition style (used for definitions and examples)
\newtheorem{definition}[dummy]{Definition}
\newtheorem{beispiel}[dummy]{Example}

\theoremstyle{plain} % Define theorem styles here based on the plain style (used for theorems, lemmas, propositions)
\newtheorem{theorem}[dummy]{Theorem}
\newtheorem{lemma}[dummy]{Lemma}
\newtheorem{corollary}[dummy]{Corollary}
\newtheorem{proposition}[dummy]{Proposition}
\newtheorem{conjecture}[dummy]{Conjecture}
\newtheorem{claim}[dummy]{Claim}

\theoremstyle{remark} % Define theorem styles here based on the remark style (used for remarks and notes)
\newtheorem{remark}[dummy]{Remark}

%----------------------------------------------------------------------------------------
%	HYPERLINKS
%---------------------------------------------------------------------------------------

\hypersetup{
%draft, % Uncomment to remove all links (useful for printing in black and white)
colorlinks=true, breaklinks=true, bookmarks=true,bookmarksnumbered,
urlcolor=webbrown, linkcolor=RoyalBlue, citecolor=webgreen, % Link colors
pdftitle={}, % PDF title
pdfauthor={\textcopyright}, % PDF Author
pdfsubject={}, % PDF Subject
pdfkeywords={}, % PDF Keywords
pdfcreator={pdfLaTeX}, % PDF Creator
pdfproducer={LaTeX with hyperref and ClassicThesis} % PDF producer
}


%----------------------------------------------------------------------------------------
%	LISTINGS
%---------------------------------------------------------------------------------------


% ********************************************************************
% listings
% ********************************************************************

\definecolor{lightergray}{gray}{0.99}

\lstset{language=[LaTeX]Tex,
    keywordstyle=\color{RoyalBlue},
    basicstyle=\normalfont\ttfamily,
    commentstyle=\color{Emerald}\ttfamily,
    stringstyle=\rmfamily,
    numbers=left,
    numberstyle=\scriptsize,
    stepnumber=5,
    numbersep=8pt,
    showstringspaces=false,
    breaklines=true,
    frameround=ftff,
    frame=lines,
    backgroundcolor=\color{lightergray}
}

\lstset{	morekeywords=%
        {RequirePackage,newboolean,DeclareOption,setboolean,%
        ProcessOptions,PackageError,ifthenelse,boolean,%
        chapterNumber,sodef,textls,allcapsspacing,%
        MakeTextLowercase,orgtheindex,endtheindex,%
        @ifpackageloaded,undefined,sfdefault,%
        DeclareRobustCommand,spacedallcaps,%
        microtypesetup,MakeTextUppercase,lowsmallcapsspacing,%
        lowsmallcapsspacing,spacedlowsmallcaps,
        spacedlowsmallcaps,lehead,headmark,color,%
        headfont,partname,thepart,titleformat,part,
        titlerule,chapter,thechapter,thesection,%
        subsection,thesubsection,thesubsubsection,%
        paragraph,theparagraph,descriptionlabel,titlespacing,%
        graffito,lineskiplimit,finalhyphendemerits,%
        colorbox,captionsetup,labelitemi,%
        myincludegraphics,hypersetup,setlength,%
        definecolor,lsstyle,textssc,subsubsection,%
        graffito@setup,includegraphics,ifdefined,%
        myTitle,textcopyright,myName,lstset,lstnewenvironment,%
        setkeys,lst@BeginAlsoWriteFile,contentsname,%
        toc@heading,@ppljLaTeX,z@,check@mathfonts,%
        sf@size,ptctitle,mtctitle,stctitle,lst@intname,%
        @empty,math@fontsfalse,@ppljscTeX,@iwonaTeX,%
        @iwonascLaTeX,@ctTeX,tw@,ct@sc,@ctTeX,f@family,%
        f@shape,ct@sc,ctLaTeX,ctLaTeXe,@twoe,@sctwoe,%
        texorpdfstring,m@th,ctTeX,@mkboth,ProvidesPackage,%
        theindex,PackageInfo,PackageWarningNoLine,%
        mtifont,mtcindent,@iwonaLaTeX,@ppljTeX,@iwonascTeX,%
        rohead,orgendtheindex,@ppljscLaTeX,%
        @ifclassloaded,toc@headingbkORrp,backreftwosep,%
        backrefalt,backreflastsep,areaset,pnumfont,%
        arsincludegraphics,ExecuteOptions,PackageWarning,textcolor,%
        MessageBreak,ars@@includegraphics,ifcld@backref,rofoot,formatchapter,%
        if@twoside},
        commentstyle=\color{Emerald}\ttfamily,%
        frame=lines}

\lstset{basicstyle=\normalfont\ttfamily}
\lstset{flexiblecolumns=true}
\lstset{moredelim={[is][\normalfont\itshape]{/*}{*/}}}
\lstset{basicstyle=\normalfont\ttfamily}
\lstset{flexiblecolumns=false}
\lstset{moredelim={[is][\ttfamily]{!?}{?!}}}
\lstset{escapeinside={£*}{*£}}
\lstset{firstnumber=last}
\lstset{moredelim={[is][\ttfamily]{!?}{?!}}}

\DeclareRobustCommand*{\pacchetto}[1]{{\normalfont\ttfamily#1}%
\index{Pacchetto!#1@\texttt{#1}}%
\index{#1@\texttt{#1}}}

\DeclareRobustCommand*{\bibtex}{\textsc{Bib}\TeX%
\index{bibtex@\textsc{Bib}\protect\TeX}%
}

\DeclareRobustCommand*{\amseuler}{\protect\AmS{} Euler%
\index{AmS Euler@\protect\AmS~Euler}%
\index{Font!AmS Euler@\protect\AmS~Euler}}

\lstnewenvironment{code}%
{\setkeys{lst}{columns=fullflexible,keepspaces=true}%
\lstset{basicstyle=\small\ttfamily}%
}{}

\lstset{extendedchars}
\lstnewenvironment{sidebyside}{%
    \global\let\lst@intname\@empty
    \setbox\z@=\hbox\bgroup
    \setkeys{lst}{columns=fullflexible,%
    linewidth=0.45\linewidth,keepspaces=true,%
    breaklines=true,%
    breakindent=0pt,%
    boxpos=t,%
    basicstyle=\small\ttfamily
}%
    \lst@BeginAlsoWriteFile{\jobname.tmp}%
}{%
    \lst@EndWriteFile\egroup
        \begin{center}%
            \begin{minipage}{0.45\linewidth}%
                \hbox to\linewidth{\box\z@\hss}
            \end{minipage}%
            \qquad
            \begin{minipage}{0.45\linewidth}%
            \setkeys{lst}{frame=none}%
                \leavevmode \catcode`\^^M=5\relax
                \small\input{\jobname.tmp}%
            \end{minipage}%
        \end{center}%
}

\newcommand{\omissis}{[\dots\negthinspace]}

\graphicspath{{Graphics/}}

\hyphenation{con-fi-gu-ra-tion}

\newcommand{\meta}[1]{$\langle${\normalfont\itshape#1}$\rangle$}
\lstset{escapeinside={£*}{*£}}


\DeclareRobustCommand*{\miktex}{MiK\TeX%
\index{miktex@MiK\protect\TeX}%
}

\DeclareRobustCommand*{\metafont}{\MF%
\index{METAFONT@\protect\MF}%
}

\DeclareRobustCommand*{\metapost}{\MP%
\index{METAPOST@\protect\MP}%
}

\DeclareRobustCommand*{\texlive}{\TeX{}~Live%
\index{texlive@\protect\TeX{}~Live}%
}

\lstset{prebreak=\raisebox{0ex}[0ex][0ex]
        {\ensuremath{\rhookswarrow}}}
\lstset{postbreak=\raisebox{0ex}[0ex][0ex]
        {\ensuremath{\rcurvearrowse\space}}}
\lstset{breaklines=true, breakatwhitespace=true}
